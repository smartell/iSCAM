%!TEX root = /Users/stevenmartell/Documents/iSCAM-project/fba/Halibut/WRITEUP/Halibut.tex

\section{Discussion} % (fold)
\label{sec:discussion}
Reducing the minimum size limit from 32 inches to 29 or 26 inches does not appear pose any substantial conservation risks.  Simulated estimates of coastwide exploitable biomass actually increase with a reduction in the minimum size limit due to reduced overall total mortality rates associated with wastage in the directed commercial fishery. Female spawning biomass is also expected to increase with reductions in the minimum size limit. If the the discard mortality rate is greater than the assumed value, this increase in exploitable biomass could be even more substantial.  This result seems counterintuitive, the general expectation would be a general decrease in spawning and exploitable biomass with decreasing size limits.  \cite{pineiii2008car} demonstrated that with increases in discard mortality rates (or a decrease in post-release survival rates), the overall spawning potential ratio (SPR) would decrease with with increasing size limits when fishing at rates equal to or greater than the maximum sustainable yield.

The overall landed value of the halibut fishery actually decreases slightly with a decrease in the minimum size limit.  With a lower size limit the size composition of the catch contains a much higher fraction of low-value 5-10 pound fish.  This result is also an artifact of the assumed \$5.00 per pound price for 5-10 pound halibut.  If this price is less, there would be even less of an economic incentive to lower the size limit.   The real potential economic benefit of lowering the size limit is associated with operational costs; with a lower size limit, the time required to catch an individuals quota could be substantially reduced because the majority of fish landed would be retained, rather than discarded.

There may also be a potential benefit from reducing the size limit if intraspecific competition for food resources is one of the factors that is related to reduced halibut growth. Retention of smaller, more abundance halibut, could potentially improve halibut growth rates by lowering the overall halibut density and improving foraging conditions \citep{walters1993recruitment,WalMart2004}.  Unfortunately the density-dependent growth model used in this simulation study is not related to annual halibut density, so it could not be used to explore this hypothesis.

One of the major caveats in this study is that the assumed coastwide selectivity curve in the commercial fishery does not change in response to changes in the size limit.  If in fact the commercial fishery selectivity did shift towards smaller sizes, then discarding of sublegal size fish would likely increase and lead to even more severe growth overfishing for this stock.  However, it is clear that there are potential economic gains to be made by reducing the minimum size limit by reducing the time required to land the quota and the operational costs.  To ensure selectivity does not change, an enforceable policy option might be to standardize fishing gear in the directed fishery.  For example, limits on hook size, hook spacing, or other tactics could be out in place to prevent a massive shift in selectivity and increase the risk of growth over fishing.  Alternatively, individual accountability for all mortality could be assigned to the individual quota holder.  In this case, say 17\% of the discarded halibut of sublegal size would count against the quota.  The latter option would almost certainly create the appropriate behavioural incentives to shift away from small halibut, but would also require 100\% observer coverage or electronic monitoring.

Lastly, even if the current minimum size limit of 32 inches is kept in place, then target fishing mortality rates need to be adjusted on an annual basis to ensure that the current fishing rate policy is commensurate with the spawning biomass reference points associated with changing growth rates.  If alternative size limits were adopted, the target fishing mortality rate would almost certainly have to be reduced to ensure it is commensurate with the SB$_{30\%}$ and SB$_{20\%}$   spawning biomass reference points assuming constant growth.  If, however, there is some persistent transition to lower or higher growth rates, then the corresponding absolute spawning biomass reference points would also decrease with lower growth rates, or increase with higher growth rates.  If halibut growth rates increase, then fishing mortality reference points that correspond to B$_{30\%}$ would also increase, and vice-versa.
% section discussion (end)



