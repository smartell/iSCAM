%!TEX root = /Users/stevenmartell/Documents/iSCAM-project/fba/Halibut/WRITEUP/Halibut.tex
\section*{Executive Summary} % (fold)
\label{sec:executive_summary}
\addcontentsline{toc}{section}{Executive Summary}

This simulation study examines two potential policy issues for the Pacific halibut fishery: Part 1 explores the potential impacts of reducing halibut bycatch in the Bering Sea and Gulf of Alaska, and Part 2 examines the potential impacts of reducing the minimum size limit  in the directed commercial fishery.  A sex- age-structured  simulation model was developed to account for the dynamics of numbers-at-age, by sex, and biomass of the coastwide population of Pacific halibut.  The simulation model was parameterized based on the recent 2011 IPHC stock assessment model and uses model estimates of the numbers-at-age, recruitment, natural mortality, fishing mortality and selectivity to initialize the simulation model from 1996--2011. Three alternative future recruitment scenarios, density-independent and density-dependent growth models are used to simulate a range of alternative scenarios 15 years into the future.  Simulation model outputs include, estimates of coastwide exploitable biomass, spawning biomass, commercial yield, discards, wastage, and the value of the directed fishery based on halibut prices in Homer Alaska.  The average exploitable biomass, spawning biomass, landed value, or other performance measures, between the years 2020--2025 was used as a summary statistic to compare alternative policy options.

Reducing non-directed fishery bycatch by 50\% in the Bering Sea Aleutian Island (BSAI) or the Gulf of Alaska (GOA) had very little impact on the simulated coastwide estimates of exploitable biomass, or spawning biomass.   The levels of bycatch reduction were redistributed to the directed fishery; about 90\% of the reduced bycatch was recovered by the commercial fishery assuming the same 2011 coastwide selectivities from the 2011 IPHC assessment.  Yield loss ratios in the directed commercial fishery were mainly less than 1; the current age/size composition of the stock and the selectivity of the commercial and bycatch gears determine the yield loss ratio.  The largest source of mortality in the coastwide stock is the directed commercial fishery.


Reducing the size limit from the current 32 inches to 29 or 26 inches, resulted in an increase in simulated estimates of exploitable biomass.  This increase was associated with a reduction in the mortality associated with the commercial wastage.  In the simulations, impacts of other users (bycatch, recreational, and personal use) of the halibut resource was assumed  constant based on the 2011 harvest values.  Decreasing the size limits lowers the overall coastwide landed value because the composition of the catch has a much higher proportion of small low value halibut (assuming \$5.00 per pound for halibut in the 5-10 pound size category).  The real economic gains to be made in the directed fishery are associated with reduced cost of fishing because fewer sublegal sized fish are discard.  Expected proportions of fish caught that are of sublegal size are 60\% with a 32 inch size limit and 9\% with a 26 inch size limit.

% section executive_summary (end)

\section*{Acknowledgments} % (fold)
\label{sec:acknowledgments}
\addcontentsline{toc}{section}{Acknowledgements}

Financial support for this study was provided by the At-sea Processors Association, United Catcher Boats, Pacific Seafood Processors Association, Alaska Groundfish Data Bank, Marine Conservation Alliance, Groundfish Forum and the Alaska Whitefish Trawlers. Thanks to Ed Richardson, Merrick Burden, and Robyn Forrest for reading over earlier drafts of this report.

I would also like to thank Bruce Leaman, Steven Hare, Juan Valero, Ray Webster, Gregg Williams and Lara Erikson from the International Pacific Halibut Commission for providing an office to work in, data, code and discussion about bycatch and size limits in the Halibut fishery.  


% section acknowledgments (end)