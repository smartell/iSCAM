%!TEX root = /Users/stevenmartell/Documents/iSCAM-project/fba/Halibut/WRITEUP/Halibut.tex

\section{Discussion} % (fold)
\label{sec:discussion}

The overarching objective of this study was to investigate the impacts of bycatch reduction in the BSAI and Gulf of Alaska on the halibut yields, exploitable biomass, spawing biomass and wastage in the directed commercial fishery.  This was accomplished by using a sex/age-structured simulation model to account for future biomass and fishing mortality rates under alternative hypotheses about future recruitment and growth rates of halibut.  The simulation model was, in part, parameterized using estimates of numbers-at-age and sex in the 1996, age-1 recruits from 1996--2006, empirical length-at-age data from the setline survey, a length-weight relationship from a recent study and fishing mortality rates from the directed fishery, 032, U32, recreational and personal use fishing fleets.  All of these parameter inputs were taken from the most recent IPHC assessment of Pacific halibut \citep[see][wobblesq model]{Hare2012Rara}.  The simulation model did not perfectly replicate estimates of exploitable biomass in the IPHC assessment largely due to the differences in the average weight-at-age data.  


The IPHC assessment model uses empirical weight-at-age data obtained from the commercial fishery catch.  At ages 6-10 the mean weight-at-age data samples are largely biased towards faster growing (larger) fish that are of legal size.  For the purposes of simulating future biomass, it was not possible to come up with a simple procedure to replicate this size selective process.  In lieu, growth curves for female and male halibut were constructed from the empirical length-at-age data obtained in the setline survey between 1996--2011.  Simulated weight-at-age data was then based on the allometric length-weight relationship developed by \cite{courcellesre}.  The net result of using this growth curve is that simulated exploitable biomass between 1996-2011 was scaled downwards.  The overall trends between the biomass simulated in this study and the IPHC assessment were nearly identical.  This difference in projected biomass would change the overall scale of the simulated results, but would have very little influence on the relative change estimated exploitable biomass (and spawning biomass) over the two alternative management procedures that involve reducing the bycatch of non-targeted fisheries in the BSAI, or the Gulf of Alaska.


There are alternative approaches to modelling density dependent growth.  In the case adopted in this model, growth rates of individual cohorts are established at birth and are strictly a function of the density of that cohort relative to the average cohort density.  The reason for adopting this approach, rather than a time-varying approach, is that it conveniently does not allow for individual fish to shrink in length.  Unfortunately, this assumption does not allow for growth rates of individual cohorts to change in response to changing environmental conditions (if they were also modelled) or changes in the density of cohorts associated with fishing.  For example, it may be plausible that growth rates of an individual cohort may increase over time as the density of halibut is reduced through natural and fishing mortality rates.  Growth rate responses to changes in density have been observed in many experimental populations of rainbow trout in freshwater lakes \citep{post1999density}. 

The results of the bycatch reductions in the BSAI and GOA regions do not appear to have much of an influence on the coastwide estimates of exploitable biomass and spawning biomass.  The principle reason is that for every pound of reduced bycatch, there is a corresponding increase in the directed fishery.  However, it appears that the directed fishery has more of an impact on the exploitable biomass than the bycatch fishery.  This was demonstrated by the ratio of lost yield in the directed fishery per pound of bycatch taken by other fisheries.  Or in other words, 10 pounds of bycatch removed is roughly equivalent to 9 pounds of yield lost to the commercial fishery. 

Another important point about bycatch impacts on the halibut stocks lies in the small regional scale.  In both this simulation model and the assessment model developed by the IPCH, there is no explicit  or implicit spatial representation of the large-scale management areas.  Unfortunately, it is not possible to examine how reducing bycath in area 4CDE, would affect the exploitable biomass, spawning biomass, wastage, etc.  in the specific areas.  Migration and movement of halibut between the management areas, and the lack of information about migration,  is one of the primary reasons why the coastwide assessment model was adopted.  It is possible that a reduction in bycatch in a specific area, may provide a local increase in exploitable biomass and impact catch rates in the directed fishery.  But at this time data are insufficient to capture these small scale dynamics.

In summary, reducing halibut bycatch by 50\% in the BSAI or GOA regions by 2.7 million pounds has no large impacts on the projected estimates of coastwide spawning biomass or exploitable biomass.  Further, this reduction of 2.7 million pounds results in about a 2.5 million pound increase in the directed fisher; simulated yield loss ratios were less than 1.0 and are a function of the current age-structure in the population.  The directed commercial fishery is by far the largest component of total mortality in the coastwide assessment model; information is lacking to determine the impacts of various fisheries at smaller spatial scales.

% section discussion (end)
