%!TEX root = /Users/stevenmartell/Documents/iSCAM-project/fba/Halibut/WRITEUP/Halibut.tex

\section{Discussion} % (fold)
\label{sec:discussion}

There are alternative approaches to modelling density dependent growth.  In the case adopted in this model, growth rates of individual cohorts are established at birth and are strictly a function of the density of that cohort relative to the average cohort density.  The reason for adopting this approach, rather than a time-varying approach, is that it conveniently does not allow for individual fish to shrink in length.  Unfortunately, this assumption does not allow for growth rates of individual cohorts to change in response to changing environmental conditions (if they were also modelled) or changes in the density of cohorts associated with fishing.  For example, it may be plausible that growth rates of an individual cohort may increase over time as the density of halibut is reduced through natural and fishing mortality rates.  Growth rate responses to changes in density have been observed in many experimental populations of rainbow trout in freshwater lakes \citep{post1999density}. 


% section discussion (end)
