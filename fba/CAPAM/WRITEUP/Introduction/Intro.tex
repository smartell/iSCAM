%!TEX root=../Selex.tex

\section*{Introduction} % (fold)
\label{sec:introduction}
There are many reasons why fisheries selectivity may vary over time and the impact of ignoring changes in selectivity in age- or size-structured stock assessment models leads to biased estimates of abundance and mortality rates.  Moreover, not accounting for changes in selectivity can lead to extremely optimistic projections in stock abundance \citep[e.g., 2J3KL cod stocks,][]{walters1996lessons}. 

Many statistical catch-age models assume age-based selectivity when in fact the underlying harvesting process is size-based. This is a reasonable assumption if fish of a given size maps to a corresponding age; however, when this approach is taken changes in size-at-age associated with changes growth rates can have serious implications on the interpretation of age-based selectivity. Changing to length-based selectivity and using empirical length-at-age data can resolve some of the model misspecification; however, ontogentic movement of fish can also lead to changes in age-based selectivity when the distribution of fishing effort, or fish distribution relative to effort, changes over time.  Recently, the International Pacific Halibut Commission (IPHC) changed from using time-invariant size-based selectivity to time-varying size-based selectivity to account for both ontogeny and the changes in the relative stock distribution 
\citep{stewart2012assessment}.  The change led to marked improvements in retrospective performance and a trend in estimated spawning biomass that was consistent with trends in survey data.  The previous assessment model was unable to consistently match the age-composition information and survey trends due to this model misspecification.

There are two general approaches for incorporating time-varying selectivity in stock assessment models; 1) the use of discrete time-blocks, and 2) continuous penalized random walk approach.  The use of discrete time-blocks should be done \emph{a priori}, where the specified time blocks represent periods of consistent fishing practice, and a new block is specified when significant changes in fishing practice occur that may result in changes in selectivity. This approach is difficult to implement.  Scientists are not necessarily qualified to identify breaks associated with changes in fishing behavior, and breaks in the terminal year are not identifiable in the model due to confounding with other model parameters. In practice, however, the time-blocks are also implemented \emph{post hoc} to rectify residual patterns in age- or size-composition data. This practice is often highly subjective.  Another discrete approach is to decompose the fisheries catch statistics into specific time periods that correspond to major transitions in fishing practice.  For example, the BC herring fishery prior to 1970 was largely a reduction fishery where herring were harvested during the winter months using purse seines.  After the collapse of the fishery in 1969, the fishery re-opened as a gill-net fishery targeting older sexually mature female herring for valuable roe.  This change in fishing practice led to a significant change in the selectivity of the fishing gear.  In some cases this can be reconciled by separating fishing fleets in the model as well.

The alternative approach is to allow for continuous changes in selectivity and model estimated selectivity parameters as a penalized random walk. In this case, specification of the variance parameter in how quickly selectivity is allowed to change is also subjective.  It should also be noted that the choice of a time-invariant selectivity is also a subjective structural assumption of the assessment model, and this choice can also greatly influence model results, estimates of reference points, and result in bias forecasts.



Changes in fisheries selectivity also has implications for reference points based on maximum sustainable yield \citep[MSY,][]{beverton1993dynamics}.  Trends towards catching smaller fish result in reductions in the harvest rate that would achieve MSY; therefore, it is important to account for changes in selectivity (and the associated uncertainty) when developing harvest policy for any given stock.

The over-arching objective is to evaluate the relative performance of assuming more or less structural complexity in selectivity when the data are in fact simple and when the data come from a fishery with dynamic changes in selectivity.  In this paper, we conduct a series of simulation experiments using a factorial design with fixed selectivity, discrete changes in selectivity, and continuous changes in selectivity and compare statistical fit, retrospective bias, and estimated policy parameters using simulated data. We also explore the use of two-dimensional interpolation methods to reduce the number of estimated latent variables when selectivity is assumed to vary over time.


 % Simulations are based on population parameters and growth information from the Pacific hake assessment conducted in 2010.  Model details, and data can be obtained from \cite{Martell2008pam}.  Model selection criterion (Deviance Information Criterion) is used to determine the effective number of estimated parameters in each case and the relative probability of choosing the correct model.  In all scenarios explored, a minimum of seven age-specific selectivity coefficients were estimated in fixed selectivity scenarios, and up to 231 selectivity coefficients were estimated in the time-varying selectivity scenarios.  

% section introduction (end)