\documentclass[table]{beamer}

%% STEP 1) SELECT THE THEM YOU WISH TO USE

% \usetheme{Goettingen} % My favorite!
% usetheme{PaloAlto}
%\usetheme{Hannover}
%\usetheme{Madrid} 
\usetheme{Boadilla} % Pretty neat, soft color.
% \usetheme{default}
% \usetheme{Warsaw}
% \usetheme{Bergen} % This template has nagivation on the left
%\usetheme{Frankfurt} % Similar to the default 
%with an extra region at the top.
% \usecolortheme{seahorse} % Simple and clean template
%\usetheme{Darmstadt} % not so good
% Uncomment the following line if you want %
% page numbers and using Warsaw theme%
\setbeamertemplate{footline}[page number]
%\setbeamercovered{transparent}
\setbeamercovered{invisible}
% To remove the navigation symbols from 
% the bottom of slides%
\setbeamertemplate{navigation symbols}{} 
%
\usepackage{graphicx}

%\usepackage{bm}         % For typesetting bold math (not \mathbold)
%\logo{\includegraphics[height=0.6cm]{yourlogo.eps}}
%

%Custom Color theme
% \usepackage{color}
% \definecolor{bottomcolour}{rgb}{0.32,0.3,0.38}
% \definecolor{middlecolour}{rgb}{0.08,0.08,0.16}
% \setbeamerfont{title}{size=\Huge}
% \setbeamercolor{structure}{fg=white}
% \setbeamertemplate{frametitle}[default][center]

% \setbeamercolor{normal text}{bg=black, fg=white}
% \setbeamertemplate{background canvas}[vertical shading]
% [bottom=bottomcolour, middle=middlecolour, top=black]

% \setbeamertemplate{itemize item}{\lower3pt\hbox{\Large\textbullet}}
% \setbeamerfont{frametitle}{size=\huge}


%end of color theme

%iscam logo
\newcommand{\iscam}{
	\raisebox{0.75ex}{$i$}%
	\textcolor{red}{\raisebox{0.25ex}{S}}%
	\textcolor{green}{\raisebox{0.00ex}{C}}%
	\textcolor{blue}{\raisebox{-.25ex}{A}}%
	\raisebox{-.50ex}{M}%
	}
% \logo{\iscam}
%end of iscam logo

%Table of contents at begining of each section
\AtBeginSection[]
{
   \begin{frame}
       \frametitle{Outline}
       \tableofcontents[currentsection]
   \end{frame}
}
%end of table of contents

\title[IPHC]{Best Practices for Modeling Time-Varying Selectivity}
\author[Martell \& Stewart]{Steven Martell \& Ian Stewart}
\institute[IPHC]
{
International Pacific Halibut Commission\\
\medskip
{\emph{stevem@iphc.int}}
}
\date{\today}
% \today will show current date. 
% Alternatively, you can specify a date.
%
\begin{document}
%
\begin{frame}
\titlepage
\end{frame}
%
\begin{frame}[t]\frametitle{Parts to this talk}
    
\tableofcontents

\end{frame}

\section{Motivation} % (fold)
\label{sec:motivation}

\begin{frame}
\frametitle{Motivation}

	
\begin{center}
	There are many \textbf{SUBJECTIVE} elements in stock assessment models.
\end{center}

\end{frame}
% section motivation (end)

\section{Virtual \& Synthetic methods} % (fold)
\label{sec:virtual_&_synthetic_methods}

\begin{frame}[m]\frametitle{VPA vs. SCA}

\begin{itemize}

	\item \alert<1>{Virtual Population Analysis}
		\begin{itemize}[<+->]
			\item Catch reported without error
			\item Incomplete cohorts
			\item Error propagation
		\end{itemize}
\vfill
	\item \alert<4>{Statistic Catch Age }
		\begin{itemize}[<+->]
			\item Confounding between error \& structural assumptions
			\item Seprability (year \& age effect)
			\item Large number of latent variables
		\end{itemize}
\end{itemize}
    


\end{frame}

% section virtual_&_synthetic_methods (end)

\section{Selectivity Models} % (fold)
\label{sec:Selectivity_Models}

\begin{frame}[m]\frametitle{Selectivity Models}
    
\begin{columns}
	\column{0.33\textwidth}
		\textbf{Fixed}\\
		\includegraphics[width=1.5in]{../FIGS/2b/iSCAMfig:Hake(2b):Selectivity2.png}\\
		Asymptotic or dome?\\
	
	
	\column{0.33\textwidth}
		\textbf{Discrete}\\
		\includegraphics[width=1.5in]{../FIGS/2b/iSCAMfig:Hake(2b):Selectivity1.png}\\
		Choice of time blocks?\\
	
	
	\column{0.33\textwidth}
		\textbf{Continuous}\\
		\includegraphics[width=1.5in]{../FIGS/3c/iSCAMfig:Hake(3c):Selectivity1.png}\\
		Variance on penalty?\\
	
\end{columns}
\vfill
\emph{How do we go about choosing the appropriate model?}
\end{frame}

\begin{frame}[m]\frametitle{How do we go about choosing the appropriate model?}
	\only<1>{
    \emph{Fishing epochs}\\
    \begin{center}
	    \includegraphics[width=1.in]{../FIGS/halibuthook.jpeg}
	    \includegraphics[width=1.7in]{../FIGS/circlehook.jpeg}
    \end{center}
    }
    \only<2>{
    \emph{Residual patterns}\\
    \begin{center}
		% \includegraphics[width=3in]{../FIGS/3c/iSCAMfig:Hake(3c):AgeCompResidual1.png}    
		\includegraphics[width=3in]{../FIGS/fig:AgeCompResidual.png}
    \end{center}
    }
    \only<3>{
    \emph{Retrospective performance}\\
    \begin{center}
		\includegraphics[width=3in]{../FIGS/1c/iSCAMfig:Hake(1c):RetrospectiveBiomass.png}    
    \end{center}
    }
    \only<4>{
    \centerline{Center for Independent Experts!}
    }
\end{frame}
% section Selectivity_models (end)


% |--------------------------------------------------------------------------------------|
% | SIMULATION EXPERIMENT
% |--------------------------------------------------------------------------------------|
% |
\section{Simulation Experiment} % (fold)
\label{sec:simulation_experiment}


\begin{frame}[m]\frametitle{Simulation experiment}
 \begin{center}
 	
 \begin{tabular}{r|cccc}
		\hline
		\textbf{\textbf{\underline{True states}}}
		&\multicolumn{4}{c}{\textbf{\underline{Assumed selectivity states}}}\\
		&{Fixed (a)} & {Disc. (b)} & {Cont. (c)}
		&{Bicub (d)} \\
		No. parameters&{N=156} & {N=160} & {N=385} & {N=239}\\
		Estimated No. &{N=89} & {N=93} & {N=318} & {N=172}\\
		\hline
		 {Fixed (1)}      & 1a & 1b & 1c & 1d \\
		 {Discrete (2)}   & 2a & 2b & 2c & 2d \\
		 {Continuous (3)} & 3a & 3b & 3c & 3d \\
		\hline


		\hline
		\end{tabular}
 \end{center}
\end{frame}
\subsection{Model overview} % (fold)
\label{sub:model_overview}

\begin{frame}[t]\frametitle{Model structure}
	Simulation: based on 2010 Pacific hake assessment
    \begin{itemize}
    	\item Age-structured, assume $M$ is known.
    	\item Conditioned on historical catch \& parameters fixed at MLE values.
    	\item Parameters: $B_o$, $h$, initial states, rec-devs, selectivities, F's, q, total variance.
    	\item Concentrated likelihood for age-comps \& estimate variance for survey \& recruitment deviates.
    \end{itemize}
    \vfill
    Data:
    \begin{itemize}
    	\item Historical removals.
    	\item Annual abundance index based on stationary $q$.
    	\item Survey age composition (logistic--time invariant).
    	\item Fishery age composition (selectivity: fixed, blocks, or continous).
    	\item Index observation error: $\sigma = 0.30$
    	\item Age-composition error (multivariate logistic): $\sigma = 0.30$
    	\item Process error: $\tau = 1.12$
    \end{itemize}
\end{frame}

\begin{frame}[t]\frametitle{Selectivities}
    Simulated
	\begin{center}
		\includegraphics[width=0.33\textwidth]{../FIGS/fig:simSelex1.png}
		\includegraphics[width=0.33\textwidth]{../FIGS/fig:simSelex2.png}
		\includegraphics[width=0.33\textwidth]{../FIGS/fig:simSelex3.png}
	\end{center}
	\vspace{-1cm}
	Estimated (7 knot cubic spline)
	\begin{center}
		\includegraphics[width=0.33\textwidth]{../FIGS/fig:estSelex1.pdf}\hspace{-1.4cm}
		\includegraphics[width=0.33\textwidth]{../FIGS/fig:estSelex2.pdf}\hspace{-1.4cm}
		\includegraphics[width=0.33\textwidth]{../FIGS/fig:estSelex3.pdf}\hspace{-1.4cm}
		\includegraphics[width=0.33\textwidth]{../FIGS/fig:estSelex4.pdf}		
	\end{center}
\end{frame}

\begin{frame}[m]\frametitle{Questions}
    \begin{enumerate}
    	\item Can DIC be used reliably to choose the correct selectivity model?
    	\item Retrospective performance of selectivity mis-specification?
    	\item Impact of selectivity mis-specification on reference points?
    \end{enumerate}
\end{frame}

% \begin{frame}[m]\frametitle{Algorithm}
%     For each model scenario:
% 	\begin{enumerate}
% 		\item Estimate model parameters for 2010 hake assesment.
% 		\item Simulate relative abundance and age-comps based on MLE values.
% 		\item Estimate joint posterior for simulated observations.
% 		\item Calculate DIC from 1000 posterior samples.
% 		\item Compute bias in estimated reference points.
% 		\item Compute 4-year mean retrospective bias.
% 		\item Repeat steps 2:5 at least 100 times for each scenario (9).
% 	\end{enumerate}

% \end{frame}
% subsection model_overview (end)



\subsection{Simulation results} % (fold)
\label{sub:simulation_results}
\begin{frame}[m,fragile]\frametitle{Model selection based on DIC}
    \only<1>{
    For each true state (fixed, discrete, continuous), fit 4 alternative assessment models to the data and calculate Deviance Information Criterion (DIC).
    }

    \only<2>{
    $\Delta$DIC
     \begin{center}
 	
 		\begin{tabular}{r|cccc}
			\hline
			\textbf{\textbf{\underline{True states}}}
			&\multicolumn{4}{c}{\textbf{\underline{Assumed selectivity states}}}\\
			&{Fixed (a)} & {Disc. (b)} & {Cont. (c)}
			&{Bicub (d)} \\
			\hline
			 
			 Fixed (1)     & 
			 \cellcolor{green}     \color[gray]{0.1959} 10.28 & 
			 \cellcolor[gray]{0.8} \color[gray]{0.4178} 21.92 & 
			 \cellcolor[gray]{0.8} \color[gray]{0.5000} 26.23 & 
			 \cellcolor[gray]{0.8} \color[gray]{0.0000} 0.00 \\
			 
			 Discrete (2)  & 
			 \cellcolor[gray]{0.8} \color[gray]{0.5000} 195.52 & 
			 \cellcolor{green}     \color[gray]{0.1151} 45.45 & 
			 \cellcolor[gray]{0.8} \color[gray]{0.0000} 0.00 & 
			 \cellcolor[gray]{0.8} \color[gray]{0.0069} 2.70 \\
			 
			 Continuous (3)& 
			 \cellcolor[gray]{0.8} \color[gray]{0.1017} 1.72 & 
			 \cellcolor[gray]{0.8} \color[gray]{0.5000} 8.45 & 
			 \cellcolor{green}     \color[gray]{0.1804} 3.05 & 
			 \cellcolor[gray]{0.8} \color[gray]{0.0000} 0.00 \\
			 
			\hline

			\hline
		\end{tabular}
 	\end{center}
    }

\end{frame}

\begin{frame}[m]\frametitle{Root Mean Square Error}
 \only<1>{
 	\begin{center}
 		\includegraphics[width=\textwidth]{../FIGS/fig:RMSE.png}
 	\end{center}
 }
 \only<2>{
  	RMSE 
     \begin{center}
 	
 		\begin{tabular}{r|cccc}
			\hline
			\textbf{\textbf{\underline{True states}}}
			&\multicolumn{4}{c}{\textbf{\underline{Assumed selectivity states}}}\\
			&{Fixed (a)} & {Disc. (b)} & {Cont. (c)}
			&{Bicub (d)} \\
			\hline
			Fixed (1)     & 
			\cellcolor[gray]{0.5} 1.942 & 
			\cellcolor[gray]{0.7} 1.937 & 
			\cellcolor[gray]{1.0} 1.867 & 
			\cellcolor[gray]{0.9} 1.915 \\ 

			Discrete (2)  & 
			\cellcolor[gray]{0.5} 1.979 & 
			\cellcolor[gray]{0.7} 1.949 & 
			\cellcolor[gray]{1.0} 1.868 & 
			\cellcolor[gray]{0.9} 1.918 \\ 

			Continuous (3)& 
			\cellcolor[gray]{0.5} 2.093 & 
			\cellcolor[gray]{0.7} 2.087 & 
			\cellcolor[gray]{1.0} 1.999 &
			\cellcolor[gray]{0.9} 2.053 \\
			\hline

			\hline
		\end{tabular}
 	\end{center}


 }


\end{frame}








\begin{frame}[t]\frametitle{Retrospective performance}

\only<1>{
	\begin{center}
		\includegraphics[width=\textwidth]{../FIGS/fig:retrospectiveBias.pdf}
	\end{center}
}
\only<2>{
	\begin{center}
		\includegraphics[width=\textwidth]{../FIGS/fig:retrospectiveSquid.pdf}
	\end{center}
}
\only<3>{
	Monte Carlo trials
	\begin{center}
		\includegraphics[width=\textwidth]{../FIGS/fig:RetrospectiveBiasMC.png}
	\end{center}	
}
\end{frame}

\begin{frame}[m]\frametitle{Retrospective bias statistic $\Omega$}
    \only<1>{
    Distance between average and absolute average bias:\\
    \[
    \mbox{bias} = \frac{1}{4} \sum_{t=2005}^{2009} \frac{B_t^y-B_t^{2010}}{B_t^{2010}}
    \]
    \[
    |\mbox{bias}| = \frac{1}{4} \sum_{t=2005}^{2009}\left| \frac{B_t^y-B_t^{2010}}{B_t^{2010}}\right|
    \]
    \[
    \Omega = \sqrt{\mbox{bias}^2 + |\mbox{bias}|^2 }
    \]
    }

    \only<2>{
    	$\Omega = 0$ implies no bias 
     \begin{center}
 	
 		\begin{tabular}{r|cccc}
			\hline
			\textbf{\textbf{\underline{True states}}}
			&\multicolumn{4}{c}{\textbf{\underline{Assumed selectivity states}}}\\
			&{Fixed (a)} & {Disc. (b)} & {Cont. (c)}
			&{Bicub (d)} \\
			\hline
			 
			 Fixed (1)     & 
			 \cellcolor[gray]{0.9} 15.75         &
			 \cellcolor[gray]{1.0} 15.66         &
			 \cellcolor[gray]{0.5} 16.53         &
			 \cellcolor[gray]{0.7} 16.40         \\
			 
			 Discrete (2)  & 
			 \cellcolor[gray]{0.9} 16.62         &
			 \cellcolor[gray]{0.5} 18.99         &
			 \cellcolor[gray]{0.7} 16.73         &
			 \cellcolor[gray]{1.0} 16.55         \\
			 
			 Continuous (3)& 
			 \cellcolor[gray]{1.0} 16.62         & 
			 \cellcolor[gray]{0.7} 17.47         & 
			 \cellcolor[gray]{0.9} 17.40         & 
			 \cellcolor[gray]{0.5} 17.48         \\
			 
			\hline

			\hline
		\end{tabular}
 	\end{center}
    }

\end{frame}

\begin{frame}[m]\frametitle{Bias in reference points}
    \only<1>{
    \begin{center}
    	\includegraphics[width=0.9\textwidth]{../FIGS/fig:RefPointBias.png}
    \end{center}
    }
    \only<2>{
        	$F_{\rm{MSY}}$ bias
     \begin{center}
 	
 		\begin{tabular}{r|cccc}
			\hline
			\textbf{\textbf{\underline{True states}}}
			&\multicolumn{4}{c}{\textbf{\underline{Assumed selectivity states}}}\\
			&{Fixed (a)} & {Disc. (b)} & {Cont. (c)}
			&{Bicub (d)} \\
			\hline
			 
			 Fixed (1)     & 
			 \cellcolor[gray]{0.5}	-0.054           &
			 \cellcolor[gray]{0.7}	-0.050           &
			 \cellcolor[gray]{0.9}	-0.009           &
			 \cellcolor[gray]{1.0}	0.007           \\
			 
			 Discrete (2)  & 
			 \cellcolor[gray]{0.5}	-0.155           &
			 \cellcolor[gray]{0.7}	-0.120           &
			 \cellcolor[gray]{1.0}	0.003           &
			 \cellcolor[gray]{0.9}	0.013           \\
			 
			 Continuous (3)& 
			 \cellcolor[gray]{0.5}	0.142           & 
			 \cellcolor[gray]{0.9}	0.086           & 
			 \cellcolor[gray]{1.0}	0.064           & 
			 \cellcolor[gray]{0.7}	0.100         \\
			 
			\hline

			\hline
		\end{tabular}
 	\end{center}
    }

\end{frame}

\begin{frame}[m]\frametitle{Rank scores}
    
     		\begin{tabular}{r|ccc|c}
			\hline
			\textbf{\textbf{\underline{True states}}}
			&\multicolumn{4}{c}{\textbf{\underline{Assumed selectivity states}}}\\
			&{Fixed (1)} & {Disc. (2)} & {Cont. (3)}
			&{Rank order} \\
			\hline
			 
			 DIC     & 
			 \cellcolor[gray]{1.0}	d,a,b,c           &
			 \cellcolor[gray]{1.0}	c,d,b,a           &
			 \cellcolor[gray]{1.0}	d,a,c,b           &
			 \cellcolor[gray]{1.0}	\textbf{d},c,b,a       \\
			 
			 $\Omega$  & 
			 \cellcolor[gray]{1.0}	b,a,d,c           &
			 \cellcolor[gray]{1.0}	d,a,c,b           &
			 \cellcolor[gray]{1.0}	a,c,b,d          &
			 \cellcolor[gray]{1.0}	a,\textbf{d},c,b        \\
			 
			 $F_{\rm{MSY}}$& 
			 \cellcolor[gray]{1.0}	d,c,b,a          & 
			 \cellcolor[gray]{1.0}	c,d,b,a          & 
			 \cellcolor[gray]{1.0}	c,b,d,a          & 
			 \cellcolor[gray]{1.0}	c,\textbf{d},b,a        \\
			 
			\hline

			\hline

		\end{tabular}


\end{frame}

% \begin{frame}[m]\frametitle{Model Selection}	
% 	\begin{block}
%     {Can DIC be used reliably to choose the correct selectivity model?}
	    
	    
% 	    DIC based on 1000 samples from the joint posterior.
	    
% 	    Monte carlo runs based on 24 simulated data sets per treatment.
	    
% 	\end{block}
% \end{frame}


% \begin{frame}[m]\frametitle{Spawning biomass}
% 	\only<1>{
%     \begin{center}
%     	Seed=991
%     	\includegraphics[height=0.75\textheight]{../FIGS/HakeSpawningBiomass991.png}
%     \end{center}
%     }
%     \only<2>{
%     \begin{center}
%     	Seed = 123
%     	\includegraphics[height=0.75\textheight]{../FIGS/HakeSpawningBiomass123.png}
%     \end{center}
%     }
% \end{frame}

% \begin{frame}[t]\frametitle{Effective number of parameters}
    
%     \begin{center}
% 		\includegraphics[height=0.8\textheight]{../FIGS/EffectiveNoPars.png} 
%     \end{center}

% \end{frame}

% \begin{frame}[t]\frametitle{Model selection vis-\`a-vis DIC}
    
% \begin{center}
% 		\includegraphics[height=0.8\textheight]{../FIGS/DIC.png} 
%     \end{center}

% \end{frame}

% \begin{frame}[m]\frametitle{Impacts on reference points}
    
% 	\only<1>{
% 		\begin{block}
% 	    {Impact of selectivity mis-specification on reference points?}
	    
% 	    MSY-based reference points based on MLE estimates.\\
% 	    Monte carlo runs based on 24 simulated data sets per treatment.\\
% 	    Compute $ \log_2\left(\frac{ F_{est} }{ F_{true} } \right)$\\
	    
% 		\end{block}
% 	}
% 	\only<2>{
% 		\begin{center}
% 			\includegraphics[height=0.8\textheight]{../FIGS/FmsyBmsyBias.png}
% 		\end{center}
% 	}

% \end{frame}



% \begin{frame}[m]\frametitle{Retrospective bias}
% 	\only<1>{
%     \begin{center}
%     	Seed = 991
%     	\includegraphics[height=0.75\textheight]{../FIGS/SSBretro991.png}
%     \end{center}
%     }
%     \only<2>{
%     \begin{center}
%     	Seed = 123
%     	\includegraphics[height=0.75\textheight]{../FIGS/SSBretro123.png}
%     \end{center}
%     }
%     \only<3>{
%     \[
%     	\mbox{bias} = \frac{1}{4} \sum_{t=2005}^{2009} 100 \frac{B_t^y-B_t^{2010}}{B_t^{2010}}
%     \]
%     }
%     \only<4>{
%     \begin{center}
%     	\includegraphics[height=0.75\textheight]{../FIGS/RetroBias.png}
%     \end{center}
%     }
% \end{frame}
% subsection simulation_results (end)

% section simulation_experiment (end)



\section{What I've learned so far} % (fold)
\label{sec:discussion}

\begin{frame}[t]\frametitle{What I learned from this exercise}
    
	\begin{itemize}
		\item<+-> Preferable to adopt a penalized random walk versus time blocks.
			\begin{itemize}
				\item Less retrospective bias \& less bias in $F_{\rm{MSY}}$.
			\end{itemize}
		\item<+-> Random walk models: loss of scale information (i.e., catch curves).
			\begin{itemize}
				\item Informative prior for scaling parameters may be necessary.
			\end{itemize}
		\item<+-> \alert<3>{Don't add small constants to age-comp likelihoods.}
			\begin{itemize}
				\item  Pool small proportions into adjacent year classes.
			\end{itemize}
		\item<+-> Tagging data could help resolve confounding in integrated models.
		\item<+-> Simulation test model selection criteria.
		\item<+-> Can also use 2 dimensional splines for selectivity.
		\begin{itemize}
			\item Appears to perform much better in n\"iave situations.
			\item Reduces number of estimated selectivity coefficients.
			\item More robust to smoothing penalities.
		\end{itemize}
		\item<+-> Ensure parameterization is continuous and differentiable.
			\begin{itemize}
				\item Avoid \texttt{max} function (not continous).
			\end{itemize}
	\end{itemize}

\end{frame}

\begin{frame}[t]\frametitle{2d cubic splines}
    Top = 231 and bottom = 60 selectivity parameters.
	\begin{center}
		\vspace{-0.15in}
		\includegraphics[height=0.5\textheight]{../FIGS/Selex231pars.pdf}\\
		\vspace{-0.4in}
		\includegraphics[height=0.5\textheight]{../FIGS/Selex60pars.pdf}
	\end{center}
	\vspace{-2.1in}
	See \texttt{bicubic\_spline}  in \texttt{statsLib.h}  at \url{http://admb-project.org/documentation/api/}.
\end{frame}

% section discussion (end)
%
% \begin{frame}[m]
% \frametitle{Constant selectivity perils}
% \begin{itemize}
% 	\item Structural assumption may lead to:
% 	\begin{itemize}
% 		\item bias in recruitment estimates (e.g., 2J3KL cod).
% 		\item precisely biased estimates of abundance.
% 		\item ...
% 	\end{itemize}
% 	\item Reference points are based on selectivity.
% 	\item A source of retrospective bias.
% 	\item Bias in forecast (e.g., Pacific halibut).
% \end{itemize}
% \end{frame}

% \begin{frame}[t]\frametitle{Selectivity models}
%     \begin{enumerate}
%     	\item Discrete time blocks, or
% 		\includegraphics[width=1.5in]{../FIGS/2b/iSCAMfig:Hake(2b):Selectivity1.png}
%     	\item Continuous penalized random walk.
% 		\includegraphics[height=2.2in]{../FIGS/3c/iSCAMfig:Hake(3c):Selectivity1.png}
%     \end{enumerate}
%     \begin{center}
% 		% \fcolorbox{white}{white}{\includegraphics[height=2.in]{../FIGS/2b/iSCAMfig:Hake(2b):Selectivity1.png}}
%     \end{center}

% \end{frame}
%
% \begin{frame}[fragile] % Notice the [fragile] option %
% \frametitle{Verbatim}
% \begin{example}[Putting Verbatim]
% \begin{verbatim}
% \begin{frame}
% \frametitle{Outline}
% \begin{block}
% {Why Beamer?}
% Does anybody need an introduction to Beamer?
% I don't think so.
% \end{block}
% % Extra carriage return causes problem with verbatim %
% \end{frame}\end{verbatim} 
% \end{example}
% \end{frame}
 
% \begin{frame}[fragile]  % notice the fragile option, since the body
% 			% contains a verbatim command
% Example of the \verb|\cite| command to give a reference is below:
% Example of citation using \cite{key1} follows on.
% \end{frame}
 
% \begin{frame}
% \frametitle{References}
% \footnotesize{
% \begin{thebibliography}{99}
%  \bibitem[Label1, 2010]{key1} Author's name (1987)
%  \newblock Title of the paper.
%  \newblock \emph{Journal Name} 55(4), 765 -- 799.
% \end{thebibliography}
% }
% \end{frame}
 
\begin{frame}[t]
\centerline{The End}
Acknowledgements:\\
\vfill 
IPHC, ADMB Foundation, CAPAM workshop organizers.\\
\vfill
Jim Ianelli and Dave Fournier for the \texttt{vcubicspline\_function\_array} class. 
\end{frame}
% End of slides
\end{document}