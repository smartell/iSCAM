% latex.default(TAC, file = fn, title = "Risk", longtable = FALSE,      landscape = FALSE, cgroup = NULL, n.cgroup = NULL, caption = cap,      label = paste("Table:Risk", hdr$Stock[i], sep = ""), na.blank = TRUE,      vbar = FALSE) 
%
\begin{table}[!tbp]
 \caption{Decision table for SOG where the risk 
			level represents the probability of exceeding the quantities specified
			in the headers of each column.  Three performance measures are considered:
			the probability of the spawning stock biomass in 2013 falling below the cutoff
			level,  the probability of the spawning stock biomass in 2013 declining from 2012, 
			and the probability of 2012 exploitation rate exceeding the 20\% level that is
			used in the harvest control rule. To use this table, first determine the 
			appropriate level of risk (e.g. 0.25 or 25\% chance),  then choose the appropriate
			management quantity (e.g. spawning biomass falling below the cutoff), and then read
			off the recommended catch (e.g., 51,565 tonnes).\label{Table:RiskSOG}} 
 \begin{center}
 \begin{tabular}{llll}\hline\hline
\multicolumn{1}{c}{Risk level}&\multicolumn{1}{c}{$P(SB_{2013})\textless$Cutoff}&\multicolumn{1}{c}{$P(SB_{2013}<SB_{2012})$}&\multicolumn{1}{c}{$P(U_{2012}<0.2)$}\tabularnewline
\hline
0.05&32,082&    0&32,080\tabularnewline
0.1&39,969&    0&37,840\tabularnewline
0.15&44,852&    0&41,406\tabularnewline
0.2&48,528&    0&44,091\tabularnewline
0.25&51,565&    0&46,308\tabularnewline
0.3&54,218&    0&48,246\tabularnewline
0.35&56,627&    0&50,005\tabularnewline
0.4&58,882&    0&51,651\tabularnewline
0.45&61,043&    0&53,230\tabularnewline
0.5&63,161&    0&54,777\tabularnewline
0.55&65,280&    0&56,324\tabularnewline
0.6&67,441&    0&57,902\tabularnewline
0.65&69,696&    0&59,549\tabularnewline
0.7&72,105&    0&61,308\tabularnewline
0.75&74,758&    0&63,245\tabularnewline
0.8&77,794&    0&65,463\tabularnewline
0.85&81,471&    0&68,148\tabularnewline
0.9&86,354&    0&71,714\tabularnewline
0.95&94,241&8,031&77,474\tabularnewline
\hline
\end{tabular}

\end{center}

\end{table}

