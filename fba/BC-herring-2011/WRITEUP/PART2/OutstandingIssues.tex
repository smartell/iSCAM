%!TEX root = /Users/stevenmartell/Documents/CURRENT PROJECTS/iSCAM-trunk/fba/BC-herring-2011/WRITEUP/BCHerring2011.tex
\section{Outstanding Issues}

The catch advice provided this year is based on the old 0.25\bo rule for establishing Cutoffs for each SAR.  Also, in moving towards a Sustainable Fisheries Framework and perhaps adopting the suggested MSY-based reference points, presents technical issue with regard to setting these reference points when population parameters are changing over time.  In this assessment, we have used the average weight-at-age to calculate \bo.  Then in an inconsistent manner, we subsequently use the average weight-at-age age (and fecundity-at-age) over the last 5 years to determine \bmsy and \fmsy reference points.  This outstanding issue should be examined more carefully before moving towards the SFF.

There was a strong retrospective bias for the PRD region. At this time, the sources of this bias are unknown, but likely due to two or more sources of data that contradict each-other or model misspecification.  The current bias problem suggest that biomass has typically been over-estimated in recent years.  The source of this bias should be investigated more closely.

There are a number of changes implemented in this \iscam\ modelling framework that have not been formally evaluated using statistical criterion. Model selection criterion such as Analysis of Deviance (DIC) should be used when adopting new formulations.  A number of alternative hypotheses (e.g., changes in selectivity, natural mortality rates) should be formally evaluated using model selection criterion.


